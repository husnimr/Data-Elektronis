% Options for packages loaded elsewhere
\PassOptionsToPackage{unicode}{hyperref}
\PassOptionsToPackage{hyphens}{url}
%
\documentclass[
]{article}
\usepackage{amsmath,amssymb}
\usepackage{lmodern}
\usepackage{iftex}
\ifPDFTeX
  \usepackage[T1]{fontenc}
  \usepackage[utf8]{inputenc}
  \usepackage{textcomp} % provide euro and other symbols
\else % if luatex or xetex
  \usepackage{unicode-math}
  \defaultfontfeatures{Scale=MatchLowercase}
  \defaultfontfeatures[\rmfamily]{Ligatures=TeX,Scale=1}
\fi
% Use upquote if available, for straight quotes in verbatim environments
\IfFileExists{upquote.sty}{\usepackage{upquote}}{}
\IfFileExists{microtype.sty}{% use microtype if available
  \usepackage[]{microtype}
  \UseMicrotypeSet[protrusion]{basicmath} % disable protrusion for tt fonts
}{}
\makeatletter
\@ifundefined{KOMAClassName}{% if non-KOMA class
  \IfFileExists{parskip.sty}{%
    \usepackage{parskip}
  }{% else
    \setlength{\parindent}{0pt}
    \setlength{\parskip}{6pt plus 2pt minus 1pt}}
}{% if KOMA class
  \KOMAoptions{parskip=half}}
\makeatother
\usepackage{xcolor}
\usepackage[margin=1in]{geometry}
\usepackage{graphicx}
\makeatletter
\def\maxwidth{\ifdim\Gin@nat@width>\linewidth\linewidth\else\Gin@nat@width\fi}
\def\maxheight{\ifdim\Gin@nat@height>\textheight\textheight\else\Gin@nat@height\fi}
\makeatother
% Scale images if necessary, so that they will not overflow the page
% margins by default, and it is still possible to overwrite the defaults
% using explicit options in \includegraphics[width, height, ...]{}
\setkeys{Gin}{width=\maxwidth,height=\maxheight,keepaspectratio}
% Set default figure placement to htbp
\makeatletter
\def\fps@figure{htbp}
\makeatother
\setlength{\emergencystretch}{3em} % prevent overfull lines
\providecommand{\tightlist}{%
  \setlength{\itemsep}{0pt}\setlength{\parskip}{0pt}}
\setcounter{secnumdepth}{-\maxdimen} % remove section numbering
\ifLuaTeX
  \usepackage{selnolig}  % disable illegal ligatures
\fi
\IfFileExists{bookmark.sty}{\usepackage{bookmark}}{\usepackage{hyperref}}
\IfFileExists{xurl.sty}{\usepackage{xurl}}{} % add URL line breaks if available
\urlstyle{same} % disable monospaced font for URLs
\hypersetup{
  pdftitle={Analisa state.x77},
  pdfauthor={Husni Mubarok Ramadhan},
  hidelinks,
  pdfcreator={LaTeX via pandoc}}

\title{Analisa state.x77}
\author{Husni Mubarok Ramadhan}
\date{2023-02-17}

\begin{document}
\maketitle

\{r setup, include=FALSE\} knitr::opts\_chunk\$set(echo = TRUE)

\hypertarget{analisa-dataset-state.x77}{%
\section{Analisa Dataset state.x77}\label{analisa-dataset-state.x77}}

\hypertarget{jalankan-rstudio-dan-di-r-console-atau-code-editor.-ketik-dan-jalankan-perintah-berikut.}{%
\subsection{Jalankan RStudio dan di R Console atau Code Editor. Ketik
dan jalankan perintah
berikut.}\label{jalankan-rstudio-dan-di-r-console-atau-code-editor.-ketik-dan-jalankan-perintah-berikut.}}

\{r, comment=NA\} \# Memanggil objek state.x77 state.x77

\emph{Tambahkan kode seperti dibawah ini.}

\{r, comment=NA\} state.x77 \textless- data.frame(state.x77)
str(state.x77) \# Kode di atas mengubah objek state.x77 menjadi
data.frame dan kemudian menampilkan struktur data.frame tersebut dengan
method string.

\emph{Tambahkan kode seperti dibawah ini.}

\{r, comment=NA\} attach(state.x77) Income quantile(Income)
quantile(Income, c(0.5, 0.25, 0.50))

\hypertarget{penjelasan}{%
\subsubsection{Penjelasan}\label{penjelasan}}

\begin{itemize}
\item
  attach()\newline Kode di atas melakukan attachment objek state.x77,
  sehingga variabel-variabel di dalamnya dapat dipanggil langsung.
\item
  quantile()\newline Kemudian dilakukan perhitungan quantile untuk
  variabel Income. Fungsi quantile() digunakan untuk menghitung
  persentil dari suatu vektor numerik.
\item
  quantile(variabel, c(0.5, 0.25, 0.50))\newline Pada kode di atas,
  dilakukan perhitungan persentil ke-0.25, ke-0.5 (median), dan ke-0.75
  dari vektor Income.
\end{itemize}

\emph{Tambahkan kode seperti dibawah ini.}

\{r, comment=NA\} \# baris pertama cor(state.x77){[},2:5{]} \# baris
kedua cor(state.x77{[},2:5{]})

\hypertarget{penjelasan-1}{%
\paragraph{Penjelasan :}\label{penjelasan-1}}

Kode di atas melakukan perhitungan korelasi antara variabel-variabel di
dalam state.x77.

\begin{itemize}
\item
  Pada baris pertama, seluruh variabel digunakan sebagai input, dan
  kemudian hanya kolom ke-2 sampai ke-5 yang ditampilkan.
\item
  Pada baris kedua, hanya kolom ke-2 sampai ke-5 yang digunakan sebagai
  input. Fungsi cor() digunakan untuk menghitung korelasi antar
  variabel. Hasilnya adalah matriks korelasi.
\end{itemize}

\end{document}
