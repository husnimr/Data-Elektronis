% Options for packages loaded elsewhere
\PassOptionsToPackage{unicode}{hyperref}
\PassOptionsToPackage{hyphens}{url}
%
\documentclass[
]{article}
\usepackage{amsmath,amssymb}
\usepackage{lmodern}
\usepackage{iftex}
\ifPDFTeX
  \usepackage[T1]{fontenc}
  \usepackage[utf8]{inputenc}
  \usepackage{textcomp} % provide euro and other symbols
\else % if luatex or xetex
  \usepackage{unicode-math}
  \defaultfontfeatures{Scale=MatchLowercase}
  \defaultfontfeatures[\rmfamily]{Ligatures=TeX,Scale=1}
\fi
% Use upquote if available, for straight quotes in verbatim environments
\IfFileExists{upquote.sty}{\usepackage{upquote}}{}
\IfFileExists{microtype.sty}{% use microtype if available
  \usepackage[]{microtype}
  \UseMicrotypeSet[protrusion]{basicmath} % disable protrusion for tt fonts
}{}
\makeatletter
\@ifundefined{KOMAClassName}{% if non-KOMA class
  \IfFileExists{parskip.sty}{%
    \usepackage{parskip}
  }{% else
    \setlength{\parindent}{0pt}
    \setlength{\parskip}{6pt plus 2pt minus 1pt}}
}{% if KOMA class
  \KOMAoptions{parskip=half}}
\makeatother
\usepackage{xcolor}
\usepackage[margin=1in]{geometry}
\usepackage{longtable,booktabs,array}
\usepackage{calc} % for calculating minipage widths
% Correct order of tables after \paragraph or \subparagraph
\usepackage{etoolbox}
\makeatletter
\patchcmd\longtable{\par}{\if@noskipsec\mbox{}\fi\par}{}{}
\makeatother
% Allow footnotes in longtable head/foot
\IfFileExists{footnotehyper.sty}{\usepackage{footnotehyper}}{\usepackage{footnote}}
\makesavenoteenv{longtable}
\usepackage{graphicx}
\makeatletter
\def\maxwidth{\ifdim\Gin@nat@width>\linewidth\linewidth\else\Gin@nat@width\fi}
\def\maxheight{\ifdim\Gin@nat@height>\textheight\textheight\else\Gin@nat@height\fi}
\makeatother
% Scale images if necessary, so that they will not overflow the page
% margins by default, and it is still possible to overwrite the defaults
% using explicit options in \includegraphics[width, height, ...]{}
\setkeys{Gin}{width=\maxwidth,height=\maxheight,keepaspectratio}
% Set default figure placement to htbp
\makeatletter
\def\fps@figure{htbp}
\makeatother
\setlength{\emergencystretch}{3em} % prevent overfull lines
\providecommand{\tightlist}{%
  \setlength{\itemsep}{0pt}\setlength{\parskip}{0pt}}
\setcounter{secnumdepth}{-\maxdimen} % remove section numbering
\ifLuaTeX
  \usepackage{selnolig}  % disable illegal ligatures
\fi
\IfFileExists{bookmark.sty}{\usepackage{bookmark}}{\usepackage{hyperref}}
\IfFileExists{xurl.sty}{\usepackage{xurl}}{} % add URL line breaks if available
\urlstyle{same} % disable monospaced font for URLs
\hypersetup{
  pdftitle={Pertemuan 6},
  pdfauthor={Husni Mubarok Ramadhan},
  hidelinks,
  pdfcreator={LaTeX via pandoc}}

\title{Pertemuan 6}
\author{Husni Mubarok Ramadhan}
\date{2023-03-03}

\begin{document}
\maketitle

\hypertarget{statistik-deskripsi-dengan-r}{%
\subsubsection{\texorpdfstring{\textbf{2~ Statistik deskripsi dengan
R}}{2~ Statistik deskripsi dengan R}}\label{statistik-deskripsi-dengan-r}}

Sebelum memulai dengan konsep dasar analisis data, seseorang harus
menyadari berbagai jenis data dan cara untuk mengatur data dalam file
komputer.

\hypertarget{beberapa-istilah-dasar}{%
\paragraph{\texorpdfstring{\textbf{2.1~ Beberapa istilah
dasar}}{2.1~ Beberapa istilah dasar}}\label{beberapa-istilah-dasar}}

\textbf{Populasi} -- agregat subjek (makhluk, benda, kasus, dan
sebagainya). Untuk studi tertentu, \emph{populasi target} harus
ditentukan: pada subjek mana kita akan menggeneralisasi atau menggunakan
hasilnya?

\textbf{Sampel} -- kumpulan subjek \emph{dalam penelitian}. Secara umum,
sampel harus representatif untuk populasi target.

\textbf{Observasi} -- unit studi atau \emph{subjek} atau individu.
Seringkali manusia, terkadang juga hewan, tumbuhan atau apa pun.

\textbf{Variabel} -- kualitas atau kuantitas, diukur atau dicatat untuk
setiap subjek dalam sampel (usia, jenis kelamin, tinggi badan, berat
badan, tingkat merokok, dll.).

\textbf{Dataset} -- seperangkat nilai dari semua variabel yang menarik
bagi semua individu dalam penelitian ini. Hasil numerik yang diperoleh
dari dataset akan digunakan untuk menarik kesimpulan tentang populasi
target.

\hypertarget{organisasi-data}{%
\paragraph{\texorpdfstring{\textbf{2.2 Organisasi
data}}{2.2 Organisasi data}}\label{organisasi-data}}

Himpunan data sebagian besar diatur (dan disimpan sebagai file komputer)
dalam bentuk \emph{matriks data}.

Matriks data yang mewakili jenis kelamin (1-laki-laki; 0-perempuan),
usia, tidak. anak-anak, berat (kg), dan tinggi (cm) 7 orang:

\begin{longtable}[]{@{}llllll@{}}
\toprule()
No & Jenis Kelamin & Umur & Nomor anak & Berat & Tinggi \\
\midrule()
\endhead
1 & 0 & 57 & 1 & 65 & 158 \\
2 & 1 & 70 & 3 & 100 & 175 \\
3 & 0 & 45 & 0 & 71 & 162 \\
4 & 0 & 38 & 2 & 58 & 164 \\
5 & 0 & 25 & 1 & 81 & 170 \\
6 & 1 & 50 & 4 & 68 & 172 \\
7 & 1 & 61 & 0 & 85 & 179 \\
\bottomrule()
\end{longtable}

Setiap baris matriks semacam itu mewakili satu pengamatan. Semua baris
memiliki panjang yang sama: data yang sama telah direkam untuk semua
individu. Setiap kolom mewakili satu variabel. Misalnya, Berat adalah
nama variabel, yang mewakili berat badan (dalam kg) seseorang.

\hypertarget{jenis-data}{%
\paragraph{\texorpdfstring{\textbf{2.3 Jenis
data}}{2.3 Jenis data}}\label{jenis-data}}

\hypertarget{data-numerik}{%
\paragraph{\texorpdfstring{• \textbf{Data
numerik}}{• Data numerik}}\label{data-numerik}}

-Data diskrit -- variabel hanya dapat mengambil nilai bilangan bulat (0,
1, 2 dll.)

Contoh: jumlah anak, jumlah teman

-Data kontinu -- setiap nilai bernomor nyata (seringkali dalam rentang
tertentu) adalah contoh yang mungkin: berat badan, usia

• \textbf{Data kualitatif (non-numerik, kategoris)}

-Data nominal: kategori tidak berurutan contoh: golongan darah, warna
mata

-Data ordinal atau terurut:ord ered kategori contoh: tingkat merokok,
sikap (baik-sedang-buruk)

Pengkodean numerik dari data nominal atau pesanan tidak membuat data
numerik!

\hypertarget{meringkasmenyajikan-data}{%
\paragraph{\texorpdfstring{\textbf{2.4 Meringkas/menyajikan
data}}{2.4 Meringkas/menyajikan data}}\label{meringkasmenyajikan-data}}

Data kontinu/diskrit Statistik lokasi ringkasan: rata-rata, median.
Rata-rata sampel adalah rata-rata aritmatika data. Ini dapat dihitung,
dengan menjumlahkan semua nilai data dan membagi jumlah dengan ukuran
sampel total.

\end{document}
